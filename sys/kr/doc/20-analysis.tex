\chapter{Аналитический раздел}
\label{cha:analysis}

\section{Cистемныt вызововы linux}
%	language=[x86masm]Assembler, 
%	caption={Диспетчер системных вызовов},
%	startline=499, 
%	linerange={498-520},

% /usr/src/linux/arch/x86/kernel/entry_32.S
\lstinputlisting[
	linerange={498-520},
	firstnumber=498,
	caption={Диспетчер системных вызовов},
	label=entry32.S,
	style=realcode]{entry32.S}
\dots
\lstinputlisting[
	firstnumber=598, 
	linerange={598-599},
	style=realcode]{entry32.S}


Как видно из листинга \ref{entry32.S} диспетчер системных вызовов linux достаточно прост. 
Он выполняет следующие действия:
\begin{itemize}
\item Сохраняет регистры.
\item В случае наличия трассировки переходит на соответствующую процедуру 
syscall_trace_entry.
\item Сравнивает номер системного вызова с максимальным и переходит на обработку ошибки 
в случае, если он больше максимального.
\item Вызывает функцию обработчика соответствующего системного вызова, расположенного
по адресу, хранящемуся в таблице системных вызовов syscall_table.
\item По завершению обработчика, если необходимо, вызывает процедуру трассировки.
\item Возвращает регистры к исходному состоянию, возвращаемое значение копируется в регистр eax.
\end{itemize}


